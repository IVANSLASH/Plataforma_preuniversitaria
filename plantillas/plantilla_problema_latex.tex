% ========================================
% PLANTILLA PARA GENERAR PROBLEMAS EN LATEX
% ========================================
% 
% INSTRUCCIONES:
% 1. Copia este archivo a ejercicios/[MATERIA]/[NOMBRE_PROBLEMA].tex
% 2. Completa todos los campos marcados con [COMPLETAR]
% 3. Escribe tu enunciado y solución
% 4. Ejecuta: python exportador/exportar_json.py
%
% EJEMPLO DE NOMBRE: ejercicios/algebra/exponentes_exp_006.tex
% ========================================

\begin{ejercicio}[
  id=[COMPLETAR_ID],                    % Ejemplo: ALG_EXP_006, CAL_DER_002, FIS_CIN_003
  materia=[COMPLETAR_MATERIA],          % algebra, calculo, fisica, geometria
  capitulo=[COMPLETAR_CAPITULO],        % exponentes, derivadas, cinematica, triangulos
  nivel=[COMPLETAR_NIVEL],              % basico, intermedio, avanzado
  procedencia="[COMPLETAR_ORIGEN]",     % "Examen UNI 2024", "Libro Álgebra 1", etc.
  visibilidad=true,                     % true (visible) o false (oculto)
  libros={[COMPLETAR_LIBROS]},          % {algebra_pre}, {calculo1, calculo_avanzado}
  youtube_url="[COMPLETAR_URL]",        % URL del video explicativo (opcional)
  mostrar_solucion=true,                % true (mostrar) o false (ocultar)
  libro_promocion=""                    % Libro promocional (opcional)
]
% ========================================
% ENUNCIADO DEL PROBLEMA
% ========================================
% 
% Escribe aquí el enunciado del problema. Puedes usar:
% - Fórmulas matemáticas: $x^2 + 3x + 1 = 0$
% - Fórmulas display: $$\int_0^1 x^2 dx = \frac{1}{3}$$
% - Texto en negrita: \textbf{Importante:}
% - Texto en cursiva: \textit{Nota:}
% - Listas: \begin{itemize} ... \end{itemize}
% - Imágenes: \begin{figure}[h] ... \end{figure}

[ESCRIBIR_ENUNCIADO_AQUI]

% ========================================
% IMAGENES (OPCIONAL)
% ========================================
% 
% Si necesitas incluir una imagen, usa este formato:
% 
% \begin{figure}[h]
% \centering
% \includegraphics[width=0.8\textwidth]{imagenes/[NOMBRE_IMAGEN].png}
% \caption{[DESCRIPCION_DE_LA_IMAGEN]}
% \label{fig:[ETIQUETA]}
% \end{figure}
% 
% NOTA: Guarda las imágenes en ejercicios/[MATERIA]/imagenes/

% ========================================
% DIAGRAMAS TIKZ (OPCIONAL)
% ========================================
% 
% Para diagramas geométricos, puedes usar TikZ:
% 
% \begin{center}
% \begin{tikzpicture}[scale=0.6]
% \coordinate (A) at (0,0);
% \coordinate (B) at (4,0);
% \coordinate (C) at (2,3);
% 
% \draw[thick] (A) -- (B) -- (C) -- cycle;
% \node[below] at (A) {$A$};
% \node[below] at (B) {$B$};
% \node[above] at (C) {$C$};
% \end{tikzpicture}
% \end{center}

\begin{solucion}
% ========================================
% SOLUCIÓN DEL PROBLEMA
% ========================================
% 
% Escribe aquí la solución paso a paso. Puedes usar:
% - Numeración: 1), 2), 3)
% - Listas: \begin{itemize} ... \end{itemize}
% - Fórmulas: $x = \frac{-b \pm \sqrt{b^2 - 4ac}}{2a}$
% - Texto destacado: \textbf{Respuesta:}
% - Notas: \textbf{Nota:}

[ESCRIBIR_SOLUCION_AQUI]

% ========================================
% RESPUESTA FINAL
% ========================================
% 
% Siempre incluye la respuesta final de manera clara:
% 
% \textbf{Respuesta:} [RESPUESTA_FINAL]

% ========================================
% NOTAS ADICIONALES (OPCIONAL)
% ========================================
% 
% \textbf{Nota:} [EXPLICACION_ADICIONAL_O_CONSEJO]

\end{solucion}
\end{ejercicio}

% ========================================
% EJEMPLOS DE USO
% ========================================

% EJEMPLO 1: PROBLEMA BÁSICO DE ÁLGEBRA
% ----------------------------------------
% \begin{ejercicio}[
%   id=ALG_EXP_006,
%   materia=algebra,
%   capitulo=exponentes,
%   nivel=basico,
%   procedencia="Examen UNI 2024",
%   visibilidad=true,
%   libros={algebra_pre},
%   youtube_url="https://www.youtube.com/watch?v=ejemplo",
%   mostrar_solucion=true,
%   libro_promocion=""
% ]
% Calcula el valor de: $2^3 \cdot 2^4 \div 2^2$
% 
% \begin{solucion}
% Aplicando las propiedades de exponentes:
% 
% 1) $2^3 \cdot 2^4 = 2^{3+4} = 2^7$
% 2) $2^7 \div 2^2 = 2^{7-2} = 2^5 = 32$
% 
% \textbf{Respuesta:} 32
% \end{solucion}
% \end{ejercicio}

% EJEMPLO 2: PROBLEMA CON IMAGEN
% ----------------------------------------
% \begin{ejercicio}[
%   id=GEO_TRI_007,
%   materia=geometria,
%   capitulo=triangulos,
%   nivel=intermedio,
%   procedencia="Libro Geometría Avanzada",
%   visibilidad=true,
%   libros={geometria_avanzado},
%   youtube_url="",
%   mostrar_solucion=true,
%   libro_promocion=""
% ]
% En el triángulo mostrado en la figura, determina el valor de $x$.
% 
% \begin{figure}[h]
% \centering
% \includegraphics[width=0.8\textwidth]{imagenes/triangulo_007.png}
% \caption{Triángulo con ángulos marcados}
% \label{fig:triangulo}
% \end{figure}
% 
% \begin{solucion}
% Por la propiedad de la suma de ángulos internos:
% 
% $x + 45° + 60° = 180°$
% $x = 180° - 45° - 60° = 75°$
% 
% \textbf{Respuesta:} $x = 75°$
% \end{solucion}
% \end{ejercicio}

% ========================================
% COMANDOS ÚTILES
% ========================================

% MATEMÁTICAS:
% ------------
% Fracciones: $\frac{a}{b}$
% Exponentes: $x^2$, $x^{n+1}$
% Raíces: $\sqrt{x}$, $\sqrt[n]{x}$
% Integrales: $\int_a^b f(x) dx$
% Derivadas: $\frac{d}{dx}f(x)$
% Límites: $\lim_{x \to a} f(x)$
% Sumatorias: $\sum_{i=1}^n x_i$
% Productorias: $\prod_{i=1}^n x_i$

% GEOMETRÍA:
% ----------
% Ángulos: $\angle ABC$
% Triángulos: $\triangle ABC$
% Paralelo: $\parallel$
% Perpendicular: $\perp$
% Congruente: $\cong$
% Similar: $\sim$
% Grados: $45^\circ$

% FÍSICA:
% --------
% Velocidad: $v = \frac{d}{t}$
% Aceleración: $a = \frac{v}{t}$
% Fuerza: $F = ma$
% Energía: $E = mc^2$
% Presión: $P = \frac{F}{A}$
% Densidad: $\rho = \frac{m}{V}$

% FORMATO:
% ---------
% Negrita: \textbf{texto}
% Cursiva: \textit{texto}
% Subrayado: \underline{texto}
% Tachado: \sout{texto}
% Centrado: \begin{center} ... \end{center}

% ========================================
% ESTRUCTURA DE CARPETAS RECOMENDADA
% ========================================
%
% ejercicios/
% ├── algebra/
% │   ├── exponentes_exp_001.tex
% │   ├── exponentes_exp_002.tex
% │   ├── polinomios_pol_001.tex
% │   └── imagenes/
% │       └── grafico_polinomio_001.png
% ├── calculo/
% │   ├── derivadas_der_001.tex
% │   ├── integrales_int_001.tex
% │   └── imagenes/
% │       └── grafico_funcion_001.png
% ├── fisica/
% │   ├── cinematica_cin_001.tex
% │   ├── dinamica_din_001.tex
% │   └── imagenes/
% │       └── diagrama_fuerzas_001.png
% └── geometria/
%     ├── triangulos_tri_001.tex
%     ├── circunferencia_cir_001.tex
%     └── imagenes/
%         └── figura_geometrica_001.png 