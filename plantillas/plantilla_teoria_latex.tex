% ========================================
% PLANTILLA PARA GENERAR TEORÍA EN LATEX
% ========================================
% 
% INSTRUCCIONES:
% 1. Copia este archivo a teoria/[MATERIA]/[CAPITULO].tex
% 2. Completa todos los campos marcados con [COMPLETAR]
% 3. Escribe el contenido teórico
% 4. Ejecuta: python exportador/exportar_json.py
%
% EJEMPLO DE NOMBRE: teoria/algebra/exponentes.tex
% ========================================

\begin{teoria}[
  materia=[COMPLETAR_MATERIA],          % algebra, calculo, fisica, geometria
  capitulo=[COMPLETAR_CAPITULO],        % exponentes, derivadas, cinematica, triangulos
  nivel=[COMPLETAR_NIVEL],              % basico, intermedio, avanzado
  titulo="[COMPLETAR_TITULO]",          % "Propiedades de Exponentes"
  descripcion="[COMPLETAR_DESCRIPCION]", % "Conceptos fundamentales de exponentes"
  tiempo_estimado=[COMPLETAR_TIEMPO],   % 15, 30, 45 (en minutos)
  dificultad=[COMPLETAR_DIFICULTAD],    % 1-5 (1=fácil, 5=difícil)
  prerequisitos={[COMPLETAR_PREREQ]},   % {aritmetica_basica}, {algebra_elemental}
  objetivos={[COMPLETAR_OBJETIVOS]},    % {"Comprender propiedades", "Resolver ejercicios"}
  tags={[COMPLETAR_TAGS]}               % {exponentes, potencias, propiedades}
]

% ========================================
% INTRODUCCIÓN
% ========================================

\section{Introducción}

[ESCRIBIR_INTRODUCCION_AQUI]

% Ejemplo:
% Los exponentes son una herramienta fundamental en matemáticas que nos permite
% expresar de manera compacta la multiplicación repetida de un número por sí mismo.
% En este capítulo aprenderemos las propiedades básicas y avanzadas de los exponentes.

% ========================================
% CONCEPTOS FUNDAMENTALES
% ========================================

\section{Conceptos Fundamentales}

\subsection{Definición Básica}

[ESCRIBIR_DEFINICION_AQUI]

% Ejemplo:
% Si $a$ es un número real y $n$ es un entero positivo, entonces:
% $$a^n = \underbrace{a \cdot a \cdot a \cdots a}_{n \text{ veces}}$$
% 
% Donde:
% - $a$ se llama la \textbf{base}
% - $n$ se llama el \textbf{exponente}
% - $a^n$ se llama la \textbf{potencia}

\subsection{Propiedades Importantes}

[ESCRIBIR_PROPIEDADES_AQUI]

% Ejemplo:
% \textbf{Propiedad 1:} $a^m \cdot a^n = a^{m+n}$
% 
% \textbf{Propiedad 2:} $\frac{a^m}{a^n} = a^{m-n}$ (con $a \neq 0$)
% 
% \textbf{Propiedad 3:} $(a^m)^n = a^{m \cdot n}$
% 
% \textbf{Propiedad 4:} $(a \cdot b)^n = a^n \cdot b^n$

% ========================================
% FÓRMULAS PRINCIPALES
% ========================================

\section{Fórmulas Principales}

[ESCRIBIR_FORMULAS_AQUI]

% Ejemplo:
% \begin{center}
% \begin{tabular}{|c|c|}
% \hline
% \textbf{Propiedad} & \textbf{Fórmula} \\
% \hline
% Producto de potencias & $a^m \cdot a^n = a^{m+n}$ \\
% \hline
% Cociente de potencias & $\frac{a^m}{a^n} = a^{m-n}$ \\
% \hline
% Potencia de potencia & $(a^m)^n = a^{m \cdot n}$ \\
% \hline
% Potencia de producto & $(a \cdot b)^n = a^n \cdot b^n$ \\
% \hline
% Potencia de cociente & $\left(\frac{a}{b}\right)^n = \frac{a^n}{b^n}$ \\
% \hline
% Exponente cero & $a^0 = 1$ (con $a \neq 0$) \\
% \hline
% Exponente negativo & $a^{-n} = \frac{1}{a^n}$ \\
% \hline
% \end{tabular}
% \end{center}

% ========================================
% EJEMPLOS RESUELTOS
% ========================================

\section{Ejemplos Resueltos}

\subsection{Ejemplo 1: Propiedades Básicas}

[ESCRIBIR_EJEMPLO_1_AQUI]

% Ejemplo:
% \textbf{Problema:} Simplifica la expresión $2^3 \cdot 2^4 \div 2^2$
% 
% \textbf{Solución:}
% \begin{align*}
% 2^3 \cdot 2^4 \div 2^2 &= 2^3 \cdot 2^4 \cdot 2^{-2} \\
% &= 2^{3+4-2} \\
% &= 2^5 \\
% &= 32
% \end{align*}
% 
% \textbf{Respuesta:} 32

\subsection{Ejemplo 2: Exponentes Negativos}

[ESCRIBIR_EJEMPLO_2_AQUI]

% ========================================
% DEMOSTRACIONES (OPCIONAL)
% ========================================

\section{Demostraciones}

[ESCRIBIR_DEMOSTRACIONES_AQUI]

% Ejemplo:
% \textbf{Demostración de la Propiedad 1:} $a^m \cdot a^n = a^{m+n}$
% 
% Por definición:
% \begin{align*}
% a^m \cdot a^n &= \underbrace{a \cdot a \cdots a}_{m \text{ veces}} \cdot \underbrace{a \cdot a \cdots a}_{n \text{ veces}} \\
% &= \underbrace{a \cdot a \cdots a}_{m+n \text{ veces}} \\
% &= a^{m+n}
% \end{align*}

% ========================================
% APLICACIONES PRÁCTICAS
% ========================================

\section{Aplicaciones Prácticas}

[ESCRIBIR_APLICACIONES_AQUI]

% Ejemplo:
% Los exponentes tienen numerosas aplicaciones en:
% 
% \begin{itemize}
% \item \textbf{Crecimiento poblacional:} $P(t) = P_0 \cdot (1+r)^t$
% \item \textbf{Interés compuesto:} $A = P \cdot (1+\frac{r}{n})^{nt}$
% \item \textbf{Decaimiento radioactivo:} $N(t) = N_0 \cdot e^{-\lambda t}$
% \item \textbf{Escalas logarítmicas:} pH, decibelios, magnitud de terremotos
% \end{itemize}

% ========================================
% ERRORES COMUNES
% ========================================

\section{Errores Comunes}

[ESCRIBIR_ERRORES_COMUNES_AQUI]

% Ejemplo:
% \textbf{Error 1:} Confundir $(a+b)^2$ con $a^2 + b^2$
% 
% \textbf{Correcto:} $(a+b)^2 = a^2 + 2ab + b^2$
% 
% \textbf{Error 2:} Olvidar que $a^0 = 1$ solo cuando $a \neq 0$
% 
% \textbf{Correcto:} $0^0$ es una forma indeterminada

% ========================================
% CONSEJOS DE ESTUDIO
% ========================================

\section{Consejos de Estudio}

[ESCRIBIR_CONSEJOS_AQUI]

% Ejemplo:
% \begin{enumerate}
% \item \textbf{Practica con números pequeños:} Comienza con bases como 2, 3, 5
% \item \textbf{Memoriza las propiedades básicas:} Son fundamentales para simplificar
% \item \textbf{Verifica tus respuestas:} Usa la calculadora para comprobar
% \item \textbf{Resuelve muchos ejercicios:} La práctica es la clave del dominio
% \end{enumerate}

% ========================================
% EJERCICIOS DE PRÁCTICA
% ========================================

\section{Ejercicios de Práctica}

[ESCRIBIR_EJERCICIOS_AQUI]

% Ejemplo:
% \textbf{Ejercicio 1:} Simplifica $3^2 \cdot 3^3 \cdot 3^{-1}$
% 
% \textbf{Ejercicio 2:} Calcula $\left(\frac{2}{3}\right)^{-2}$
% 
% \textbf{Ejercicio 3:} Resuelve la ecuación $2^x = 8$
% 
% \textbf{Ejercicio 4:} Simplifica $\frac{x^5 \cdot y^3}{x^2 \cdot y^4}$

% ========================================
% RESUMEN
% ========================================

\section{Resumen}

[ESCRIBIR_RESUMEN_AQUI]

% Ejemplo:
% En este capítulo hemos aprendido:
% 
% \begin{itemize}
% \item Las propiedades fundamentales de los exponentes
% \item Cómo simplificar expresiones con potencias
% \item Aplicaciones prácticas en la vida real
% \item Errores comunes a evitar
% \end{itemize}
% 
% Estas herramientas son esenciales para el estudio de álgebra avanzada,
% cálculo y otras ramas de las matemáticas.

% ========================================
% REFERENCIAS Y ENLACES
% ========================================

\section{Referencias y Enlaces}

[ESCRIBIR_REFERENCIAS_AQUI]

% Ejemplo:
% \begin{itemize}
% \item \textbf{Libro recomendado:} "Álgebra Elemental" por Charles P. McKeague
% \item \textbf{Videos:} Canal de YouTube "Matemáticas Fáciles"
% \item \textbf{Páginas web:} Khan Academy - Exponentes
% \item \textbf{Ejercicios adicionales:} Problemas del libro de texto
% \end{itemize}

\end{teoria}

% ========================================
% EJEMPLOS DE USO
% ========================================

% EJEMPLO 1: TEORÍA BÁSICA DE EXPONENTES
% ----------------------------------------
% \begin{teoria}[
%   materia=algebra,
%   capitulo=exponentes,
%   nivel=basico,
%   titulo="Propiedades de Exponentes",
%   descripcion="Conceptos fundamentales de exponentes y sus propiedades",
%   tiempo_estimado=30,
%   dificultad=2,
%   prerequisitos={aritmetica_basica},
%   objetivos={"Comprender propiedades básicas", "Simplificar expresiones"},
%   tags={exponentes, potencias, propiedades}
% ]
% 
% \section{Introducción}
% Los exponentes son una herramienta fundamental...
% 
% \section{Conceptos Fundamentales}
% \subsection{Definición Básica}
% Si $a$ es un número real y $n$ es un entero positivo...
% 
% \section{Fórmulas Principales}
% \begin{center}
% \begin{tabular}{|c|c|}
% \hline
% \textbf{Propiedad} & \textbf{Fórmula} \\
% \hline
% Producto de potencias & $a^m \cdot a^n = a^{m+n}$ \\
% \hline
% \end{tabular}
% \end{center}
% 
% \section{Ejemplos Resueltos}
% \subsection{Ejemplo 1: Propiedades Básicas}
% \textbf{Problema:} Simplifica $2^3 \cdot 2^4$...
% 
% \section{Resumen}
% En este capítulo hemos aprendido...
% 
% \end{teoria}

% ========================================
% COMANDOS ÚTILES PARA TEORÍA
% ========================================

% ESTRUCTURA:
% -----------
% \section{Título de Sección}
% \subsection{Subtítulo}
% \subsubsection{Sub-subtítulo}
% \paragraph{Párrafo}
% \subparagraph{Sub-párrafo}

% LISTAS:
% -------
% \begin{itemize}
% \item Elemento 1
% \item Elemento 2
% \end{itemize}
% 
% \begin{enumerate}
% \item Elemento 1
% \item Elemento 2
% \end{enumerate}
% 
% \begin{description}
% \item[Término] Definición
% \item[Concepto] Explicación
% \end{description}

% TABLAS:
% --------
% \begin{center}
% \begin{tabular}{|c|c|}
% \hline
% Columna 1 & Columna 2 \\
% \hline
% Dato 1 & Dato 2 \\
% \hline
% \end{tabular}
% \end{center}

% ALINEACIÓN DE ECUACIONES:
% -------------------------
% \begin{align*}
% x &= a + b \\
% &= c + d \\
% &= e + f
% \end{align*}

% CASOS:
% -------
% \begin{cases}
% x + y = 5 \\
% 2x - y = 1
% \end{cases}

% MATRICES:
% ----------
% \begin{pmatrix}
% a & b \\
% c & d
% \end{pmatrix}

% ========================================
% ESTRUCTURA DE CARPETAS PARA TEORÍA
% ========================================
%
% teoria/
% ├── algebra/
% │   ├── exponentes.tex
% │   ├── polinomios.tex
% │   ├── ecuaciones.tex
% │   └── imagenes/
% │       ├── grafico_exponencial.png
% │       └── diagrama_polinomio.png
% ├── calculo/
% │   ├── derivadas.tex
% │   ├── integrales.tex
% │   ├── limites.tex
% │   └── imagenes/
% │       ├── grafico_derivada.png
% │       └── area_integral.png
% ├── fisica/
% │   ├── cinematica.tex
% │   ├── dinamica.tex
% │   ├── termodinamica.tex
% │   └── imagenes/
% │       ├── diagrama_fuerzas.png
% │       └── grafico_movimiento.png
% └── geometria/
%     ├── triangulos.tex
%     ├── circunferencia.tex
%     ├── poligonos.tex
%     └── imagenes/
%         ├── figura_triangulo.png
%         └── diagrama_circulo.png 