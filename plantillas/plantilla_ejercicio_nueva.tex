% PLANTILLA PARA CREAR EJERCICIOS - VERSIÓN NUEVA ESTRUCTURA
% ==========================================================
% 
% INSTRUCCIONES DE USO:
% 1. Copia esta plantilla para crear nuevos ejercicios
% 2. Guárdala en: ejercicios_nuevo/[MATERIA_PRINCIPAL]/[CAPITULO]/[ID].tex
% 3. Las imágenes van en: ejercicios_nuevo/[MATERIA_PRINCIPAL]/[CAPITULO]/imagenes/
%
% EJEMPLO DE UBICACIÓN:
% ejercicios_nuevo/matematicas_preuniversitaria/algebra/MATU_ALG_001.tex
% ejercicios_nuevo/fisica_preuniversitaria/cinematica/FISU_CIN_001.tex
%
% CÓDIGOS DE MATERIAS (4 caracteres):
% MATU = Matemáticas Preuniversitaria
% FISU = Física Preuniversitaria  
% QUIM = Química Preuniversitaria
% LENG = Lenguaje y Literatura
% CAL2 = Cálculo 2
% FIS1 = Física 1, etc.
%
% FORMATO ID: [MATERIA]_[CAPITULO]_[NÚMERO]
% Ejemplo: MATU_ALG_001, FISU_CIN_042

\begin{ejercicio}[
  id=MATU_ALG_001,                    % ID único del ejercicio
  materia_principal=matematicas_preuniversitaria,  % Materia principal
  codigo_materia=MATU,                % Código de 4 caracteres
  capitulo=algebra,                   % Capítulo específico
  subtema=exponentes,                 % Subtema dentro del capítulo
  nivel=basico,                       % basico, intermedio, avanzado
  procedencia="Examen UNI 2023",      % Origen del problema
  visibilidad=web_impreso,            % web_impreso, solo_impreso, solo_web
  tiempo_estimado=5,                  % Tiempo en minutos
  libros={matematicas_pre, algebra_completa},  % Libros donde aparece
  dificultad=2,                       % Escala 1-5
  tags={exponentes, propiedades}      % Etiquetas para búsqueda
]

% ==========================================
% ENUNCIADO DEL PROBLEMA
% ==========================================
% Escribe aquí el enunciado de tu problema.
% Usa LaTeX para matemáticas: $x^2$, $\frac{a}{b}$, $\sqrt{x}$
% 
% Para incluir imágenes:
% \begin{figure}[h]
%   \includegraphics[width=0.6\textwidth]{imagenes/nombre_imagen.png}
%   \caption{Descripción de la imagen}
% \end{figure}

Simplifica la siguiente expresión aplicando las propiedades de los exponentes:

$$\frac{2^5 \cdot 2^3}{2^4} \cdot 2^{-1}$$

a) $2^3$  \quad b) $2^4$  \quad c) $2^2$  \quad d) $2^5$  \quad e) $2^1$

\begin{solucion}
% ==========================================
% SOLUCIÓN DETALLADA
% ==========================================
% Escribe aquí la solución paso a paso.
% Usa LaTeX para matemáticas y fórmulas.
% 
% Formato recomendado:
% 1. Identificar propiedades a usar
% 2. Aplicar paso a paso
% 3. Resultado final

\textbf{Propiedades a usar:}
\begin{itemize}
    \item Producto de potencias: $a^m \cdot a^n = a^{m+n}$
    \item Cociente de potencias: $\frac{a^m}{a^n} = a^{m-n}$
\end{itemize}

\textbf{Desarrollo:}

Paso 1: Simplificar el numerador
$$2^5 \cdot 2^3 = 2^{5+3} = 2^8$$

Paso 2: Simplificar la fracción
$$\frac{2^8}{2^4} = 2^{8-4} = 2^4$$

Paso 3: Multiplicar por el factor restante
$$2^4 \cdot 2^{-1} = 2^{4+(-1)} = 2^{4-1} = 2^3$$

\textbf{Respuesta:} a) $2^3$

\textbf{Nota:} También se puede resolver directamente:
$$\frac{2^5 \cdot 2^3}{2^4} \cdot 2^{-1} = 2^{5+3-4-1} = 2^{8-5} = 2^3$$
\end{solucion}
\end{ejercicio}

% ==========================================
% NOTAS ADICIONALES PARA EL AUTOR
% ==========================================
%
% NIVELES DE DIFICULTAD:
% - basico: Conceptos fundamentales, aplicación directa
% - intermedio: Requiere varios pasos, conexión de conceptos  
% - avanzado: Problemas complejos, pensamiento crítico
%
% VISIBILIDAD:
% - web_impreso: Aparece en ambos formatos
% - solo_impreso: Solo en libros físicos (soluciones exclusivas)
% - solo_web: Solo en plataforma digital
%
% TIEMPO ESTIMADO:
% Tiempo promedio que toma un estudiante resolver el problema
%
% TAGS:
% Palabras clave para filtros y búsquedas avanzadas