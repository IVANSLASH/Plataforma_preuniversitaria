% PLANTILLA PARA CREAR EJERCICIOS
% ===============================
% Copia esta plantilla y modifica según tus necesidades
% Guarda como: ejercicios/[MATERIA]/[CAPITULO]_[CODIGO].tex
%
% FORMATOS DE ID DISPONIBLES:
% - 3 dígitos: ALG_EXP_001 (hasta 999 problemas)
% - 4 dígitos: ALG_EXP_0001 (hasta 9,999 problemas)

\begin{ejercicio}[
  id=ALG_EXP_0001,  % Puedes usar ALG_EXP_001 o ALG_EXP_0001
  materia=algebra,
  capitulo=exponentes,
  nivel=basico,
  procedencia="Examen UNI 2023",
  visibilidad=true,
  libros={algebra_pre, algebra_basica}
]
% ENUNCIADO DEL PROBLEMA
% Escribe aquí el enunciado de tu problema.
% Usa LaTeX para matemáticas: $x^2$, $\frac{a}{b}$, $\sqrt{x}$

Calcula el valor de la siguiente expresión:

$$2^3 \cdot 2^4 \div 2^2$$

\begin{solucion}
% SOLUCIÓN DETALLADA
% Escribe aquí la solución paso a paso.
% Usa LaTeX para matemáticas y fórmulas.

Para resolver esta expresión, aplicamos las propiedades de los exponentes:

1) \textbf{Producto de potencias de igual base}: $a^m \cdot a^n = a^{m+n}$
2) \textbf{Cociente de potencias de igual base}: $a^m \div a^n = a^{m-n}$

Aplicando estas propiedades:

$$2^3 \cdot 2^4 \div 2^2 = 2^{3+4} \div 2^2 = 2^7 \div 2^2 = 2^{7-2} = 2^5 = 32$$

\textbf{Respuesta:} 32
\end{solucion}
\end{ejercicio} 