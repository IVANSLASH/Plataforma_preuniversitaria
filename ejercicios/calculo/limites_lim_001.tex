\begin{ejercicio}[
  id=CAL_LIM_001,
  materia=calculo,
  capitulo=limites,
  nivel=basico,
  procedencia="Examen CEPRE 2023",
  visibilidad=true,
  libros={calculo1, calculo_basico},
  youtube_url="",
  mostrar_solucion=true,
  libro_promocion=""
]
Calcula el siguiente límite:

$$\lim_{x \to 2} \frac{x^2 - 4}{x - 2}$$

\begin{solucion}
Este es un límite de la forma $\frac{0}{0}$ (indeterminación). Vamos a resolverlo:

1) **Factorizamos el numerador**:
   $$x^2 - 4 = (x + 2)(x - 2)$$

2) **Simplificamos la fracción**:
   $$\frac{x^2 - 4}{x - 2} = \frac{(x + 2)(x - 2)}{x - 2} = x + 2$$

3) **Calculamos el límite**:
   $$\lim_{x \to 2} (x + 2) = 2 + 2 = 4$$

\textbf{Respuesta:} 4

\textbf{Nota:} Este límite se puede resolver también aplicando la regla de L'Hôpital, pero la factorización es más directa.
\end{solucion}
\end{ejercicio} 