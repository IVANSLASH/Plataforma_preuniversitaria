\begin{ejercicio}[
  id=FIS_MOV_001,
  materia=fisica,
  capitulo=movimiento,
  nivel=intermedio,
  procedencia="Examen CEPRE 2023",
  visibilidad=true,
  libros={fisica_basica, fisica_mecanica},
  youtube_url="https://www.youtube.com/watch?v=ejemplo_fisica",
  mostrar_solucion=true,
  libro_promocion=""
]
Un bloque de masa $m = 2$ kg se encuentra sobre una superficie horizontal rugosa. Se aplica una fuerza horizontal $F = 10$ N al bloque. Si el coeficiente de fricción cinética es $\mu_k = 0.3$, determina la aceleración del bloque.

\textbf{Nota:} Observa el diagrama de fuerzas para entender mejor el problema.

\begin{figure}[h]
\centering
\includegraphics[width=0.8\textwidth]{imagenes/diagrama_fuerzas_001.png}
\caption{Diagrama de fuerzas actuando sobre el bloque}
\label{fig:diagrama_fuerzas}
\end{figure}

\begin{solucion}
Para resolver este problema, analizamos las fuerzas que actúan sobre el bloque:

1) \textbf{Fuerzas verticales:}
   \begin{itemize}
   \item Peso: $W = mg = 2 \cdot 9.8 = 19.6$ N (hacia abajo)
   \item Normal: $N = W = 19.6$ N (hacia arriba)
   \end{itemize}

2) \textbf{Fuerzas horizontales:}
   \begin{itemize}
   \item Fuerza aplicada: $F = 10$ N (hacia la derecha)
   \item Fricción cinética: $f_k = \mu_k N = 0.3 \cdot 19.6 = 5.88$ N (hacia la izquierda)
   \end{itemize}

3) \textbf{Fuerza neta horizontal:}
   $$F_{net} = F - f_k = 10 - 5.88 = 4.12 \text{ N}$$

4) \textbf{Aceleración:}
   $$a = \frac{F_{net}}{m} = \frac{4.12}{2} = 2.06 \text{ m/s}^2$$

\textbf{Respuesta:} La aceleración del bloque es $2.06$ m/s².

\textbf{Nota:} El diagrama muestra claramente cómo las fuerzas se equilibran verticalmente y la fuerza neta horizontal produce la aceleración.
\end{solucion}
\end{ejercicio} 