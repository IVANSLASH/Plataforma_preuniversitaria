\documentclass[12pt,a4paper]{book}
\usepackage[utf8]{inputenc}
\usepackage[spanish]{babel}
\usepackage{amsmath,amssymb,amsfonts}
\usepackage{geometry}
\usepackage{graphicx}
\usepackage{hyperref}
\usepackage{xcolor}
\usepackage{fancyhdr}
\usepackage{titlesec}

% Configuración de página
\geometry{margin=2.5cm}
\pagestyle{fancy}
\fancyhf{}
\fancyhead[L]{\leftmark}
\fancyhead[R]{\thepage}
\renewcommand{\headrulewidth}{0.4pt}

% Configuración de títulos
\titleformat{\chapter}[display]
{\normalfont\huge\bfseries}{\chaptertitlename\ \thechapter}{20pt}{\Huge}

% Definición del entorno ejercicio
\newenvironment{ejercicio}[1][]{%
    \begin{trivlist}\item[\hskip \labelsep {\bfseries Ejercicio.}]
    \if\relax\detokenize{#1}\relax
    \else
        \textbf{[#1]}
    \fi
}{%
    \end{trivlist}
}

% Definición del entorno solución
\newenvironment{solucion}{%
    \begin{trivlist}\item[\hskip \labelsep {\bfseries Solución.}]
    \color{blue}
}{%
    \color{black}
    \end{trivlist}
}

% Información del documento
\title{\Huge \textbf{Álgebra Preuniversitaria}\\[0.5cm]
       \Large Teoría y Ejercicios Resueltos}
\author{Plataforma Preuniversitaria}
\date{\today}

\begin{document}

\maketitle

\tableofcontents
\newpage

\chapter{Exponentes y Radicales}

\section{Propiedades de los Exponentes}

Los exponentes son una herramienta fundamental en álgebra. A continuación presentamos las propiedades más importantes:

\begin{enumerate}
    \item $a^m \cdot a^n = a^{m+n}$ (Producto de potencias de igual base)
    \item $\frac{a^m}{a^n} = a^{m-n}$ (Cociente de potencias de igual base)
    \item $(a^m)^n = a^{m \cdot n}$ (Potencia de una potencia)
    \item $(a \cdot b)^n = a^n \cdot b^n$ (Potencia de un producto)
    \item $\left(\frac{a}{b}\right)^n = \frac{a^n}{b^n}$ (Potencia de un cociente)
    \item $a^0 = 1$ (Exponente cero)
    \item $a^{-n} = \frac{1}{a^n}$ (Exponente negativo)
\end{enumerate}

\section{Ejercicios Resueltos}

% Incluir ejercicios desde el repositorio central
% Estos ejercicios se pueden incluir manualmente o mediante scripts

% Ejercicio básico de exponentes
\begin{ejercicio}[ALG\_EXP\_001]
Calcula el valor de la siguiente expresión:

$$2^3 \cdot 2^4 \div 2^2$$

\begin{solucion}
Para resolver esta expresión, aplicamos las propiedades de los exponentes:

1) \textbf{Producto de potencias de igual base}: $a^m \cdot a^n = a^{m+n}$
2) \textbf{Cociente de potencias de igual base}: $a^m \div a^n = a^{m-n}$

Aplicando estas propiedades:

$$2^3 \cdot 2^4 \div 2^2 = 2^{3+4} \div 2^2 = 2^7 \div 2^2 = 2^{7-2} = 2^5 = 32$$

\textbf{Respuesta:} 32
\end{solucion}
\end{ejercicio}

% Ejercicio intermedio de exponentes
\begin{ejercicio}[ALG\_EXP\_002]
Simplifica la siguiente expresión:

$$\left(\frac{x^2 y^{-3}}{x^{-1} y^2}\right)^{-2} \cdot \frac{x^4}{y^6}$$

\begin{solucion}
Vamos a simplificar paso a paso:

1) \textbf{Primero simplificamos la fracción dentro del paréntesis}:
   $$\frac{x^2 y^{-3}}{x^{-1} y^2} = x^{2-(-1)} \cdot y^{-3-2} = x^3 \cdot y^{-5}$$

2) \textbf{Aplicamos el exponente -2}:
   $$\left(x^3 \cdot y^{-5}\right)^{-2} = x^{3 \cdot (-2)} \cdot y^{-5 \cdot (-2)} = x^{-6} \cdot y^{10}$$

3) \textbf{Multiplicamos por la segunda fracción}:
   $$x^{-6} \cdot y^{10} \cdot \frac{x^4}{y^6} = x^{-6+4} \cdot y^{10-6} = x^{-2} \cdot y^4$$

4) \textbf{Expresamos con exponentes positivos}:
   $$x^{-2} \cdot y^4 = \frac{y^4}{x^2}$$

\textbf{Respuesta:} $\frac{y^4}{x^2}$
\end{solucion}
\end{ejercicio}

% Ejercicio intermedio de exponentes con valores numéricos
\begin{ejercicio}[ALG\_EXP\_003]
Si $x = 2$ e $y = 3$, calcula el valor de la siguiente expresión:

$$\left(\frac{x^4 \cdot y^{-2}}{x^{-1} \cdot y^3}\right)^2 \cdot \frac{x^6}{y^4}$$

\begin{solucion}
Vamos a resolver este problema paso a paso:

1) \textbf{Primero simplificamos la fracción dentro del paréntesis}:
   $$\frac{x^4 \cdot y^{-2}}{x^{-1} \cdot y^3} = x^{4-(-1)} \cdot y^{-2-3} = x^5 \cdot y^{-5}$$

2) \textbf{Aplicamos el exponente 2}:
   $$\left(x^5 \cdot y^{-5}\right)^2 = x^{5 \cdot 2} \cdot y^{-5 \cdot 2} = x^{10} \cdot y^{-10}$$

3) \textbf{Multiplicamos por la segunda fracción}:
   $$x^{10} \cdot y^{-10} \cdot \frac{x^6}{y^4} = x^{10+6} \cdot y^{-10-4} = x^{16} \cdot y^{-14}$$

4) \textbf{Expresamos con exponentes positivos}:
   $$x^{16} \cdot y^{-14} = \frac{x^{16}}{y^{14}}$$

5) \textbf{Sustituimos los valores} $x = 2$ e $y = 3$:
   $$\frac{2^{16}}{3^{14}} = \frac{65536}{4782969} = \frac{65536}{4782969}$$

\textbf{Respuesta:} $\frac{65536}{4782969}$

\textbf{Nota:} Este resultado se puede simplificar, pero se deja en forma de fracción para mayor precisión.
\end{solucion}
\end{ejercicio}

% Ejercicio intermedio de exponentes con simplificación compleja
\begin{ejercicio}[ALG\_EXP\_004]
Simplifica la siguiente expresión y encuentra el valor cuando $a = 4$ y $b = 2$:

$$\frac{(a^3 b^{-2})^4 \cdot (a^{-1} b^3)^2}{(a^2 b^{-1})^3 \cdot (a^{-2} b^2)^2}$$

\begin{solucion}
Vamos a simplificar esta expresión compleja paso a paso:

1) \textbf{Desarrollamos las potencias en el numerador}:
   $$(a^3 b^{-2})^4 = a^{3 \cdot 4} \cdot b^{-2 \cdot 4} = a^{12} \cdot b^{-8}$$
   $$(a^{-1} b^3)^2 = a^{-1 \cdot 2} \cdot b^{3 \cdot 2} = a^{-2} \cdot b^6$$

2) \textbf{Desarrollamos las potencias en el denominador}:
   $$(a^2 b^{-1})^3 = a^{2 \cdot 3} \cdot b^{-1 \cdot 3} = a^6 \cdot b^{-3}$$
   $$(a^{-2} b^2)^2 = a^{-2 \cdot 2} \cdot b^{2 \cdot 2} = a^{-4} \cdot b^4$$

3) \textbf{Reescribimos la expresión original}:
   $$\frac{a^{12} \cdot b^{-8} \cdot a^{-2} \cdot b^6}{a^6 \cdot b^{-3} \cdot a^{-4} \cdot b^4}$$

4) \textbf{Agrupamos términos semejantes}:
   $$\frac{a^{12-2} \cdot b^{-8+6}}{a^{6-4} \cdot b^{-3+4}} = \frac{a^{10} \cdot b^{-2}}{a^2 \cdot b^1}$$

5) \textbf{Simplificamos aplicando propiedades de división}:
   $$\frac{a^{10}}{a^2} \cdot \frac{b^{-2}}{b^1} = a^{10-2} \cdot b^{-2-1} = a^8 \cdot b^{-3}$$

6) \textbf{Expresamos con exponentes positivos}:
   $$a^8 \cdot b^{-3} = \frac{a^8}{b^3}$$

7) \textbf{Sustituimos los valores} $a = 4$ y $b = 2$:
   $$\frac{4^8}{2^3} = \frac{65536}{8} = 8192$$

\textbf{Respuesta:} 8192

\textbf{Nota:} La expresión simplificada es $\frac{a^8}{b^3}$, que es mucho más manejable que la expresión original.
\end{solucion}
\end{ejercicio}

\section{Ejercicios Propuestos}

% Aquí se pueden incluir ejercicios adicionales
% que no necesariamente están en el repositorio central

\begin{ejercicio}
Calcula: $3^2 \cdot 3^5 \cdot 3^{-3}$
\end{ejercicio}

\begin{ejercicio}
Simplifica: $\frac{(x^3)^2 \cdot x^4}{x^8}$
\end{ejercicio}

\chapter{Límites y Continuidad}

\section{Concepto de Límite}

El límite de una función $f(x)$ cuando $x$ tiende a $a$ es el valor al que se aproxima $f(x)$ cuando $x$ se acerca arbitrariamente a $a$.

$$\lim_{x \to a} f(x) = L$$

\section{Ejercicios Resueltos}

% Ejercicio de límites
\begin{ejercicio}[CAL\_LIM\_001]
Calcula el siguiente límite:

$$\lim_{x \to 2} \frac{x^2 - 4}{x - 2}$$

\begin{solucion}
Este es un límite de la forma $\frac{0}{0}$ (indeterminación). Vamos a resolverlo:

1) \textbf{Factorizamos el numerador}:
   $$x^2 - 4 = (x + 2)(x - 2)$$

2) \textbf{Simplificamos la fracción}:
   $$\frac{x^2 - 4}{x - 2} = \frac{(x + 2)(x - 2)}{x - 2} = x + 2$$

3) \textbf{Calculamos el límite}:
   $$\lim_{x \to 2} (x + 2) = 2 + 2 = 4$$

\textbf{Respuesta:} 4

\textbf{Nota:} Este límite se puede resolver también aplicando la regla de L'Hôpital, pero la factorización es más directa.
\end{solucion}
\end{ejercicio}

\appendix

\chapter{Soluciones de Ejercicios Propuestos}

\section{Soluciones del Capítulo 1}

\begin{solucion}
Para $3^2 \cdot 3^5 \cdot 3^{-3}$:

$$3^2 \cdot 3^5 \cdot 3^{-3} = 3^{2+5-3} = 3^4 = 81$$
\end{solucion}

\begin{solucion}
Para $\frac{(x^3)^2 \cdot x^4}{x^8}$:

$$\frac{(x^3)^2 \cdot x^4}{x^8} = \frac{x^6 \cdot x^4}{x^8} = \frac{x^{10}}{x^8} = x^2$$
\end{solucion}

\end{document} 