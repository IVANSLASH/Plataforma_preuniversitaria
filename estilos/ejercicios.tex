% Estilos para ejercicios de la Plataforma Preuniversitaria
% ========================================================
% Este archivo define los entornos y estilos para ejercicios
% Autor: Plataforma Preuniversitaria
% Fecha: 2024

% Paquetes necesarios
\RequirePackage{amsmath,amssymb,amsfonts}
\RequirePackage{xcolor}
\RequirePackage{framed}
\RequirePackage{tcolorbox}
\RequirePackage{enumitem}
\RequirePackage{geometry}

% Configuración de colores
\definecolor{ejercicioColor}{RGB}{41, 128, 185}    % Azul
\definecolor{solucionColor}{RGB}{39, 174, 96}      % Verde
\definecolor{advertenciaColor}{RGB}{230, 126, 34}  % Naranja
\definecolor{errorColor}{RGB}{231, 76, 60}         % Rojo

% Configuración de márgenes para ejercicios
\geometry{
    left=2.5cm,
    right=2.5cm,
    top=2.5cm,
    bottom=2.5cm
}

% Entorno para ejercicios
\newtcolorbox{ejercicio}[1][]{
    enhanced,
    breakable,
    colback=white,
    colframe=ejercicioColor,
    coltitle=white,
    fonttitle=\bfseries,
    title=Ejercicio,
    boxrule=2pt,
    arc=8pt,
    left=15pt,
    right=15pt,
    top=10pt,
    bottom=10pt,
    before skip=15pt,
    after skip=15pt,
    #1
}

% Entorno para soluciones
\newtcolorbox{solucion}{
    enhanced,
    breakable,
    colback=white,
    colframe=solucionColor,
    coltitle=white,
    fonttitle=\bfseries,
    title=Solución,
    boxrule=2pt,
    arc=8pt,
    left=15pt,
    right=15pt,
    top=10pt,
    bottom=10pt,
    before skip=10pt,
    after skip=10pt
}

% Entorno para ejercicios numerados
\newcounter{ejercicioCounter}
\newtcolorbox{ejercicioNumerado}[1][]{
    enhanced,
    breakable,
    colback=white,
    colframe=ejercicioColor,
    coltitle=white,
    fonttitle=\bfseries,
    title=Ejercicio~\theejercicioCounter,
    boxrule=2pt,
    arc=8pt,
    left=15pt,
    right=15pt,
    top=10pt,
    bottom=10pt,
    before skip=15pt,
    after skip=15pt,
    #1
}

% Comando para incrementar contador de ejercicios
\newcommand{\nuevoejercicio}{
    \stepcounter{ejercicioCounter}
}

% Entorno para ejercicios con metadatos
\newenvironment{ejercicioConMetadatos}[1][]{
    \begin{ejercicio}[title=Ejercicio~#1]
}{
    \end{ejercicio}
}

% Entorno para soluciones detalladas
\newenvironment{solucionDetallada}{
    \begin{solucion}
    \textbf{Solución detallada:}
    \vspace{5pt}
}{
    \end{solucion}
}

% Entorno para pasos de solución
\newenvironment{pasosSolucion}{
    \begin{enumerate}[label=\textbf{Paso \arabic*:}, leftmargin=*]
}{
    \end{enumerate}
}

% Entorno para respuestas
\newtcolorbox{respuesta}{
    enhanced,
    colback=yellow!10,
    colframe=yellow!50,
    coltitle=black,
    fonttitle=\bfseries,
    title=Respuesta,
    boxrule=1pt,
    arc=5pt,
    left=10pt,
    right=10pt,
    top=8pt,
    bottom=8pt
}

% Entorno para notas importantes
\newtcolorbox{notaImportante}{
    enhanced,
    colback=orange!10,
    colframe=advertenciaColor,
    coltitle=white,
    fonttitle=\bfseries,
    title=Nota Importante,
    boxrule=1pt,
    arc=5pt,
    left=10pt,
    right=10pt,
    top=8pt,
    bottom=8pt
}

% Entorno para definiciones
\newtcolorbox{definicion}{
    enhanced,
    colback=blue!10,
    colframe=blue!50,
    coltitle=white,
    fonttitle=\bfseries,
    title=Definición,
    boxrule=1pt,
    arc=5pt,
    left=10pt,
    right=10pt,
    top=8pt,
    bottom=8pt
}

% Entorno para teoremas
\newtcolorbox{teorema}{
    enhanced,
    colback=green!10,
    colframe=green!50,
    coltitle=white,
    fonttitle=\bfseries,
    title=Teorema,
    boxrule=1pt,
    arc=5pt,
    left=10pt,
    right=10pt,
    top=8pt,
    bottom=8pt
}

% Comandos útiles para ejercicios
\newcommand{\ejercicioId}[1]{\textbf{ID:} #1}
\newcommand{\ejercicioMateria}[1]{\textbf{Materia:} #1}
\newcommand{\ejercicioNivel}[1]{\textbf{Nivel:} #1}
\newcommand{\ejercicioCapitulo}[1]{\textbf{Capítulo:} #1}
\newcommand{\ejercicioProcedencia}[1]{\textbf{Procedencia:} #1}

% Comando para mostrar metadatos del ejercicio
\newcommand{\mostrarMetadatos}[5]{
    \begin{center}
    \small
    \begin{tabular}{ll}
        \ejercicioId{#1} & \ejercicioMateria{#2} \\
        \ejercicioNivel{#3} & \ejercicioCapitulo{#4} \\
        \multicolumn{2}{l}{\ejercicioProcedencia{#5}}
    \end{tabular}
    \end{center}
}

% Configuración para listas de ejercicios
\setlist[enumerate,1]{label=\textbf{\arabic*.}, leftmargin=*}
\setlist[enumerate,2]{label=\textbf{(\alph*)}, leftmargin=*}

% Configuración para fórmulas matemáticas
\everymath{\displaystyle}

% Comandos para espacios en ejercicios
\newcommand{\espacioEjercicio}{\vspace{1cm}}
\newcommand{\espacioSolucion}{\vspace{0.5cm}}

% Comando para separador de ejercicios
\newcommand{\separadorEjercicios}{
    \vspace{1cm}
    \hrule
    \vspace{1cm}
}

% Configuración para encabezados de página
\pagestyle{fancy}
\fancyhf{}
\fancyhead[L]{\leftmark}
\fancyhead[R]{\thepage}
\renewcommand{\headrulewidth}{0.4pt}

% Configuración para pies de página
\fancyfoot[C]{
    \small
    Plataforma Preuniversitaria - Ejercicios
}

% Comando para título de sección de ejercicios
\newcommand{\tituloSeccionEjercicios}[1]{
    \section*{#1}
    \vspace{0.5cm}
}

% Comando para título de capítulo de ejercicios
\newcommand{\tituloCapituloEjercicios}[1]{
    \chapter*{#1}
    \vspace{1cm}
}

% Configuración para referencias cruzadas
\newcommand{\refEjercicio}[1]{\textbf{Ejercicio~\ref{#1}}}
\newcommand{\refSolucion}[1]{\textbf{Solución~\ref{#1}}}

% Comando para índice de ejercicios
\newcommand{\indiceEjercicios}{
    \tableofcontents
    \newpage
}

% Configuración para ejercicios con imágenes
\newcommand{\ejercicioConImagen}[2]{
    \begin{ejercicio}
    #1
    \begin{center}
    \includegraphics[width=0.8\textwidth]{#2}
    \end{center}
    \end{ejercicio}
}

% Comando para ejercicios de opción múltiple
\newenvironment{ejercicioOpcionMultiple}{
    \begin{ejercicio}
    \begin{enumerate}[label=\textbf{\Alph*.}, leftmargin=*]
}{
    \end{enumerate}
    \end{ejercicio}
}

% Comando para ejercicios de verdadero/falso
\newenvironment{ejercicioVerdaderoFalso}{
    \begin{ejercicio}
    \textbf{Indica si cada afirmación es verdadera (V) o falsa (F):}
    \begin{enumerate}[label=\textbf{\arabic*.}, leftmargin=*]
}{
    \end{enumerate}
    \end{ejercicio}
}

% Configuración para ejercicios de desarrollo
\newenvironment{ejercicioDesarrollo}{
    \begin{ejercicio}
    \textbf{Desarrolla paso a paso:}
}{
    \end{ejercicio}
}

% Comando para ejercicios de completar
\newcommand{\ejercicioCompletar}[1]{
    \begin{ejercicio}
    \textbf{Completa los espacios en blanco:}
    #1
    \end{ejercicio}
}

% Configuración para ejercicios de demostración
\newenvironment{ejercicioDemostracion}{
    \begin{ejercicio}
    \textbf{Demuestra:}
}{
    \end{ejercicio}
}

% Comando para ejercicios de cálculo
\newcommand{\ejercicioCalculo}[1]{
    \begin{ejercicio}
    \textbf{Calcula:}
    \[#1\]
    \end{ejercicio}
}

% Comando para ejercicios de simplificación
\newcommand{\ejercicioSimplificar}[1]{
    \begin{ejercicio}
    \textbf{Simplifica:}
    \[#1\]
    \end{ejercicio}
}

% Comando para ejercicios de factorización
\newcommand{\ejercicioFactorizar}[1]{
    \begin{ejercicio}
    \textbf{Factoriza:}
    \[#1\]
    \end{ejercicio}
}

% Configuración para ejercicios de resolución de ecuaciones
\newenvironment{ejercicioEcuacion}{
    \begin{ejercicio}
    \textbf{Resuelve la ecuación:}
}{
    \end{ejercicio}
}

% Comando para ejercicios de sistemas de ecuaciones
\newcommand{\ejercicioSistemaEcuaciones}[1]{
    \begin{ejercicio}
    \textbf{Resuelve el sistema de ecuaciones:}
    \[#1\]
    \end{ejercicio}
}

% Configuración para ejercicios de geometría
\newenvironment{ejercicioGeometria}{
    \begin{ejercicio}
    \textbf{Problema de geometría:}
}{
    \end{ejercicio}
}

% Comando para ejercicios de trigonometría
\newcommand{\ejercicioTrigonometria}[1]{
    \begin{ejercicio}
    \textbf{Problema de trigonometría:}
    \[#1\]
    \end{ejercicio}
}

% Configuración para ejercicios de cálculo diferencial
\newenvironment{ejercicioCalculoDiferencial}{
    \begin{ejercicio}
    \textbf{Problema de cálculo diferencial:}
}{
    \end{ejercicio}
}

% Configuración para ejercicios de cálculo integral
\newenvironment{ejercicioCalculoIntegral}{
    \begin{ejercicio}
    \textbf{Problema de cálculo integral:}
}{
    \end{ejercicio}
}

% Comando para ejercicios de límites
\newcommand{\ejercicioLimite}[1]{
    \begin{ejercicio}
    \textbf{Calcula el límite:}
    \[\lim_{#1}\]
    \end{ejercicio}
}

% Comando para ejercicios de derivadas
\newcommand{\ejercicioDerivada}[1]{
    \begin{ejercicio}
    \textbf{Calcula la derivada:}
    \[\frac{d}{dx}\left(#1\right)\]
    \end{ejercicio}
}

% Comando para ejercicios de integrales
\newcommand{\ejercicioIntegral}[1]{
    \begin{ejercicio}
    \textbf{Calcula la integral:}
    \[\int #1 \, dx\]
    \end{ejercicio}
}

% Configuración para ejercicios de física
\newenvironment{ejercicioFisica}{
    \begin{ejercicio}
    \textbf{Problema de física:}
}{
    \end{ejercicio}
}

% Configuración para ejercicios de química
\newenvironment{ejercicioQuimica}{
    \begin{ejercicio}
    \textbf{Problema de química:}
}{
    \end{ejercicio}
}

% Configuración para ejercicios de biología
\newenvironment{ejercicioBiologia}{
    \begin{ejercicio}
    \textbf{Problema de biología:}
}{
    \end{ejercicio}
}

% Configuración para ejercicios de lenguaje
\newenvironment{ejercicioLenguaje}{
    \begin{ejercicio}
    \textbf{Problema de lenguaje:}
}{
    \end{ejercicio}
}

% Comando para finalizar el documento
\newcommand{\finDocumento}{
    \vspace{2cm}
    \begin{center}
    \textbf{--- Fin de los ejercicios ---}
    \end{center}
} 