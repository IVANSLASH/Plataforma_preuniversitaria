\documentclass[12pt,a4paper]{article}
\usepackage[utf8]{inputenc}
\usepackage[spanish]{babel}
\usepackage{amsmath,amssymb,amsfonts}
\usepackage{geometry}
\usepackage{fancyhdr}
\usepackage{enumitem}

% Configuración de página
\geometry{margin=2.5cm}
\pagestyle{fancy}
\fancyhf{}
\fancyhead[L]{Simulacro de Álgebra - Exponentes Básicos}
\fancyhead[R]{Página \thepage}
\renewcommand{\headrulewidth}{0.4pt}

% Configuración de listas
\setlist[enumerate]{label=\arabic*., leftmargin=*}

% Información del documento
\title{\Huge \textbf{Simulacro de Álgebra - Exponentes Básicos}}
\author{Plataforma Preuniversitaria}
\date{\today}

\begin{document}

\maketitle

\section*{Instrucciones}
Resuelve los siguientes ejercicios. Tienes 2 horas para completar el simulacro.

\section*{Ejercicios}


\begin{enumerate}
\item[\textbf{1.}] Calcula el valor de la siguiente expresión:

$$2^3 \cdot 2^4 \div 2^2$$
\end{enumerate}

\vspace{1cm}


\begin{enumerate}
\item[\textbf{2.}] Calcula el valor de la siguiente expresión:

$$5^2 \cdot 5^3$$
\end{enumerate}

\vspace{1cm}


\newpage
\section*{Soluciones}


\textbf{1.} Para resolver esta expresión, aplicamos las propiedades de los exponentes:

1) **Producto de potencias de igual base**: $a^m \cdot a^n = a^{m+n}$
2) **Cociente de potencias de igual base**: $a^m \div a^n = a^{m-n}$

Aplicando estas propiedades:

$$2^3 \cdot 2^4 \div 2^2 = 2^{3+4} \div 2^2 = 2^7 \div 2^2 = 2^{7-2} = 2^5 = 32$$

\textbf{Respuesta:} 32

\vspace{0.5cm}


\textbf{2.} Para resolver esta expresión, aplicamos la propiedad de los exponentes:

\textbf{Propiedad:} $a^m \cdot a^n = a^{m+n}$

Aplicando esta propiedad:

$$5^2 \cdot 5^3 = 5^{2+3} = 5^5 = 3125$$

\textbf{Respuesta:} 3125

\textbf{Nota:} Este es un ejemplo usando el formato de ID con 4 dígitos (ALG\_EXP\_0001).

\vspace{0.5cm}


\end{document}
