\begin{ejercicio}[
  id=MATU_ALG_007,
  materia_principal=matematicas_preuniversitaria,
  codigo_materia=MATU,
  capitulo=algebra,
  subtema=exponentes,
  nivel=intermedio,
  procedencia="Problemas Resueltos de Matemática",
  visibilidad=web_impreso,
  tiempo_estimado=6,
  libros={matematicas_pre},
  dificultad=3,
  mostrar_solucion=true,
  tags={exponentes, fracciones, sustitucion}
]

Desarrollar:  
Si $(ab)^m = 1$ y $a^{2m} - b^{2m} = 2$, hallar el valor de $Q$:

\[
Q = \frac{a^{-m} - b^{-m}}{a^{-m} + b^{-m}} - \frac{a^{-m} + b^{-m}}{a^{-m} - b^{-m}}
\]

\begin{solucion}

\textbf{Datos del problema}
\begin{itemize}
    \item $(ab)^m=1 \ \Rightarrow\ (ab)^{2m}=1$.
    \item $a^{2m}-b^{2m}=2$.
\end{itemize}

\textbf{Cambio de variable}  
Sea $x=a^{-m}$ y $y=b^{-m}$. Entonces
\[
Q=\frac{x-y}{x+y}-\frac{x+y}{x-y}
  =\frac{(x-y)^2-(x+y)^2}{(x+y)(x-y)}.
\]

\textbf{Algebra}  
\[
(x-y)^2-(x+y)^2=(x^2-2xy+y^2)-(x^2+2xy+y^2)=-4xy,
\]
\[
(x+y)(x-y)=x^2-y^2.
\]
Por tanto
\[
Q=\frac{-4xy}{x^2-y^2}.
\]

\textbf{Volver a $a,b$}  
\[
xy=a^{-m}b^{-m}=(ab)^{-m}=\big((ab)^m\big)^{-1}=1,
\]
\[
x^2-y^2=a^{-2m}-b^{-2m}
=\frac{b^{2m}-a^{2m}}{a^{2m}b^{2m}}
=\frac{-(a^{2m}-b^{2m})}{(ab)^{2m}}
=\frac{-2}{1}=-2.
\]

\textbf{Resultado}  
\[
Q=\frac{-4\cdot 1}{-2}=2.
\]

\textbf{Respuesta final:} $\boxed{Q=2}$.

\end{solucion}

\end{ejercicio}
