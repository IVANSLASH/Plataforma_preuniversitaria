\begin{ejercicio}[
  id=MATU_ALG_005,
  materia_principal=matematicas_preuniversitaria,
  codigo_materia=MATU,
  capitulo=algebra,
  subtema=exponentes,
  nivel=intermedio,
  procedencia="Examen UNMSM 2024",
  visibilidad=web_impreso,
  tiempo_estimado=5,
  libros={algebra_pre, algebra_intermedio},
  dificultad=2,
  tags={exponentes},
  youtube_url="",
  mostrar_solucion=true,
  libro_promocion=""
]
% ENUNCIADO DEL PROBLEMA
% Escribe aquí el enunciado de tu problema.
% Usa LaTeX para matemáticas: $x^2$, $\frac{a}{b}$, $\sqrt{x}$

Simplifica la siguiente expresión y encuentra el valor cuando $a = 4$ y $b = 2$:

$$\frac{(a^3 b^{-2})^4 \cdot (a^{-1} b^3)^2}{(a^2 b^{-1})^3 \cdot (a^{-2} b^2)^2}$$

\begin{solucion}
% SOLUCIÓN DETALLADA
% Escribe aquí la solución paso a paso.
% Usa LaTeX para matemáticas y fórmulas.

Vamos a simplificar esta expresión compleja paso a paso:

1) \textbf{Desarrollamos las potencias en el numerador}:
   $$(a^3 b^{-2})^4 = a^{3 \cdot 4} \cdot b^{-2 \cdot 4} = a^{12} \cdot b^{-8}$$
   $$(a^{-1} b^3)^2 = a^{-1 \cdot 2} \cdot b^{3 \cdot 2} = a^{-2} \cdot b^6$$

2) \textbf{Desarrollamos las potencias en el denominador}:
   $$(a^2 b^{-1})^3 = a^{2 \cdot 3} \cdot b^{-1 \cdot 3} = a^6 \cdot b^{-3}$$
   $$(a^{-2} b^2)^2 = a^{-2 \cdot 2} \cdot b^{2 \cdot 2} = a^{-4} \cdot b^4$$

3) \textbf{Reescribimos la expresión original}:
   $$\frac{a^{12} \cdot b^{-8} \cdot a^{-2} \cdot b^6}{a^6 \cdot b^{-3} \cdot a^{-4} \cdot b^4}$$

4) \textbf{Agrupamos términos semejantes}:
   $$\frac{a^{12-2} \cdot b^{-8+6}}{a^{6-4} \cdot b^{-3+4}} = \frac{a^{10} \cdot b^{-2}}{a^2 \cdot b^1}$$

5) \textbf{Simplificamos aplicando propiedades de división}:
   $$\frac{a^{10}}{a^2} \cdot \frac{b^{-2}}{b^1} = a^{10-2} \cdot b^{-2-1} = a^8 \cdot b^{-3}$$

6) \textbf{Expresamos con exponentes positivos}:
   $$a^8 \cdot b^{-3} = \frac{a^8}{b^3}$$

7) \textbf{Sustituimos los valores} $a = 4$ y $b = 2$:
   $$\frac{4^8}{2^3} = \frac{65536}{8} = 8192$$

\textbf{Respuesta:} 8192

\textbf{Nota:} La expresión simplificada es $\frac{a^8}{b^3}$, que es mucho más manejable que la expresión original.
\end{solucion}
\end{ejercicio}