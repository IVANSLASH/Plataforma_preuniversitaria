\begin{ejercicio}[
  id=MATU_ALG_006,
  materia_principal=matematicas_preuniversitaria,
  codigo_materia=MATU,
  capitulo=algebra,
  subtema=exponentes,
  nivel=basico,
  procedencia="Ejemplo con formato 4 dígitos",
  visibilidad=web_impreso,
  tiempo_estimado=5,
  libros={algebra_pre, algebra_basica},
  dificultad=2,
  tags={exponentes},
  youtube_url="",
  mostrar_solucion=true,
  libro_promocion=""
]
Calcula el valor de la siguiente expresión:

$$5^2 \cdot 5^3$$

\begin{solucion}
Para resolver esta expresión, aplicamos la propiedad de los exponentes:

\textbf{Propiedad:} $a^m \cdot a^n = a^{m+n}$

Aplicando esta propiedad:

$$5^2 \cdot 5^3 = 5^{2+3} = 5^5 = 3125$$

\textbf{Respuesta:} 3125

\textbf{Nota:} Este es un ejemplo usando el formato de ID con 4 dígitos (ALG\_EXP\_0001).
\end{solucion}
\end{ejercicio}