\begin{ejercicio}[
  id=MATU_ALG_003,
  materia_principal=matematicas_preuniversitaria,
  codigo_materia=MATU,
  capitulo=algebra,
  subtema=exponentes,
  nivel=intermedio,
  procedencia="Libro Álgebra Avanzada",
  visibilidad=web_impreso,
  tiempo_estimado=5,
  libros={algebra_pre, algebra_avanzada},
  dificultad=2,
  tags={exponentes},
  youtube_url="",
  mostrar_solucion=true,
  libro_promocion=""
]
Simplifica la siguiente expresión:

$$\left(\frac{x^2 y^{-3}}{x^{-1} y^2}\right)^{-2} \cdot \frac{x^4}{y^6}$$

\begin{solucion}
Vamos a simplificar paso a paso:

1) **Primero simplificamos la fracción dentro del paréntesis**:
   $$\frac{x^2 y^{-3}}{x^{-1} y^2} = x^{2-(-1)} \cdot y^{-3-2} = x^3 \cdot y^{-5}$$

2) **Aplicamos el exponente -2**:
   $$\left(x^3 \cdot y^{-5}\right)^{-2} = x^{3 \cdot (-2)} \cdot y^{-5 \cdot (-2)} = x^{-6} \cdot y^{10}$$

3) **Multiplicamos por la segunda fracción**:
   $$x^{-6} \cdot y^{10} \cdot \frac{x^4}{y^6} = x^{-6+4} \cdot y^{10-6} = x^{-2} \cdot y^4$$

4) **Expresamos con exponentes positivos**:
   $$x^{-2} \cdot y^4 = \frac{y^4}{x^2}$$

\textbf{Respuesta:} $\frac{y^4}{x^2}$
\end{solucion}
\end{ejercicio}