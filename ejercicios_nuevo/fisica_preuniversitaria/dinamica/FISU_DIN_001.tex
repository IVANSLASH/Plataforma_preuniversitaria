\begin{ejercicio}[
  id=FISU_DIN_001,
  materia_principal=fisica_preuniversitaria,
  codigo_materia=FISU,
  capitulo=dinamica,
  subtema=segunda_ley_newton,
  nivel=intermedio,
  procedencia="Examen CEPRE 2023",
  visibilidad=web_impreso,
  tiempo_estimado=8,
  libros={fisica_basica, fisica_mecanica},
  dificultad=3,
  tags={friccion, fuerzas, aceleracion, segunda_ley_newton},
  youtube_url="https://www.youtube.com/watch?v=ejemplo_fisica",
  mostrar_solucion=true,
  libro_promocion=""
]
Un bloque de masa $m = 2$ kg se encuentra sobre una superficie horizontal rugosa. Se aplica una fuerza horizontal $F = 10$ N al bloque. Si el coeficiente de fricción cinética es $\mu_k = 0.3$, determina la aceleración del bloque.

\textbf{Dato:} $g = 9.8$ m/s²

\textbf{Nota:} Observa el diagrama de fuerzas para entender mejor el problema.

\begin{figure}[h]
\centering
\includegraphics[width=0.8\textwidth]{imagenes/diagrama_fuerzas_001.png}
\caption{Diagrama de fuerzas actuando sobre el bloque}
\label{fig:diagrama_fuerzas}
\end{figure}

\begin{solucion}
Para resolver este problema de dinámica, aplicamos la Segunda Ley de Newton analizando todas las fuerzas:

\textbf{Paso 1: Análisis de fuerzas verticales}

En el eje vertical, el bloque está en equilibrio:
\begin{itemize}
\item Peso: $W = mg = 2 \times 9.8 = 19.6$ N (hacia abajo)
\item Fuerza normal: $N = W = 19.6$ N (hacia arriba)
\end{itemize}

$$\sum F_y = 0 \Rightarrow N - W = 0 \Rightarrow N = 19.6 \text{ N}$$

\textbf{Paso 2: Análisis de fuerzas horizontales}

En el eje horizontal actúan:
\begin{itemize}
\item Fuerza aplicada: $F = 10$ N (hacia la derecha)
\item Fuerza de fricción cinética: $f_k = \mu_k N = 0.3 \times 19.6 = 5.88$ N (hacia la izquierda)
\end{itemize}

\textbf{Paso 3: Fuerza neta horizontal}
$$F_{neta} = F - f_k = 10 - 5.88 = 4.12 \text{ N}$$

\textbf{Paso 4: Aplicar Segunda Ley de Newton}
$$F_{neta} = ma$$
$$a = \frac{F_{neta}}{m} = \frac{4.12}{2} = 2.06 \text{ m/s}^2$$

\textbf{Respuesta:} La aceleración del bloque es $a = 2.06$ m/s².

\textbf{Verificación:} La aceleración es positiva, lo que confirma que el bloque se mueve en la dirección de la fuerza aplicada, como era de esperarse dado que $F > f_k$.
\end{solucion}
\end{ejercicio}